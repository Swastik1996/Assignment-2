\documentclass{article}
\usepackage{array,amsmath,graphicx,upgreek}
\usepackage{ amssymb }
\usepackage{hyperref}
\newcolumntype{@}{>{\global\let\currentrowstyle\relax}}
\newcolumntype{^}{>{\currentrowstyle}}
\newcommand{\rowstyle}[1]{\gdef\currentrowstyle{#1}%
  #1\ignorespaces
}
\graphicspath{ {} }
\usepackage[a4paper,bindingoffset=0.2in,%
            left=0.2in,right=0.25in,top=0.25in,bottom=1in,%
            footskip=.25in]{geometry}
\begin{document}
\begin{center}
\huge{\textbf{CS648 Assignment 2}}\\
\end{center}
SWASTIK SHARMA (14741)
\hfill
ANUBHAV SHRIVASTAVA (14114)

\section*{1. 2-Dimensional Pattern Matching}
1. \\
\textbf{Notations} :
Let $T$ be the text bit-string and $P$ the pattern bit-string. Size of $T$ is n and $P$ is m. \\
$M(P)$ is mapping of the pattern P to a $2 X 2$ matrix as given in the problem statement. \\
$M(P)^{-1}$ is the inverse matrix to M(P).
Let l and r be two indices in T s.t. $1 \leq l \leq r \leq n.$ \\
$T[l,r]$ is the substring starting at index l and ending at r. \\
Consequently, $M(T[l,r])$ is mapping of $T[l,r]$ to $2 X 2$ matrix as given in the problem statement. \\
\[
\hspace{-2cm}
M(\epsilon)=
	\begin{bmatrix}
	1 & 0 \\
	0 & 1 
	\end{bmatrix}
\hspace{1cm}
M(0)=
	\begin{bmatrix}
	1 & 0 \\
	1 & 1 
	\end{bmatrix}
\hspace{1cm}
M(1)=
	\begin{bmatrix}
	1 & 1 \\
	0 & 1 
	\end{bmatrix}
\hspace{1cm}
M(0)^{-1}=
	\begin{bmatrix}
	1 & 0 \\
	-1 & 1
	\end{bmatrix}
\hspace{1cm}
M(1)^{-1}=
	\begin{bmatrix}
	1 & -1 \\
	0 & 1 
	\end{bmatrix}
\]
\textbf{Algorithm} :
Basically, we will compare $M(P)$ with $M(T[i,i+m-1])$ $\forall$ i in $[1,n-m+1]$.\\
We will calculate $M(P)$ by right multiplying corresponding $M(P[i])$ for i from 1 to m to an identity matrix. Similiarly,we will calcuate $M(T[1,m])$ After that to get, $M(T[i,i+m-1])$
for i $\textgreater$ 1, we will take $M(T[i-1,i+m-2])$ , right multiply $M(T[i+m,i+m])$ and left multiply $M(T[i-1,i-1])^{-1}$ to it. \\ 
The following pseudo code formalize the above algorithm. \\
$ pattern \leftarrow M(P) ;$ \\
$ current \leftarrow M(T[1,m]) ; $ \\
$for $ $ i $ = $ 1 $ $to$ $ n - m + 1 $ \\
\hspace*{1cm}  $if$ $pattern$ $==$ $current $ \\ 
\hspace*{2cm} print "Pattern present at index i"; \\ 
\hspace*{1cm} $ if$ $i$ $\textless$ $n - m + 1 $ \\ 
\hspace*{2cm}  	current = $multiply(current,M(T[i+m,i+m]))$; \\ 
\hspace*{2cm}  	current = $multiply(M(T[i,i]),current)$; \\\\
\textbf{Order Analysis} :\\
Calculating $M(P)$ takes $O(m)$ time  \\
Calculating $M(T[1,m])$ takes $O(m)$ time \\
Loop Runs $n - m + 1$ times and at each iteration we do a $O(1)$ computation.Hence loop takes O(n - m + 1) time. \\ 
Overall Complexity = $O(n + m)$ time. \\\\
2. \\
\textbf{Notation} :  \\\\
Let M be a matrix then M mod p denotes a matrix in which all entries of M are replaced by their modulo with p.\\
Hence, We perform the above algorithm but at each step, while multiplying two matrices, we take modulo with p. \\
Let $M(A)$ and $M(B)$ be two matrices which are mapping of string $A$ and $B$ of size m. \\
Now, if $M(A)$ == $M(B)$ then $M(A)$ $mod$ $p$ == $M(A)$ $mod$ $p$ then $A$ == $B$ which is true. \\
But if $A$ != $B$ and then $M(A)$ != $M(B)$ then it might be possible that $M(A)$ $mod$ $p$ == $M(B)$ $mod$ $p$. which will imply that $A$ == $B$ which is not true. \\
Hence, this algorithm will fail iff $M(A)$ != $M(B)$ and $M(A)$ $mod$ $p$ == $M(B)$ $mod$ $p$. \\ 
Let $D = M(A) - M(B)$ \\ 
$M(A)$ $mod$ $p$ == $M(B)$ $mod$ $p$ \\
$\Longrightarrow$ $M(A) - M(B)$ $mod$ $p$ == $0$ ( where $0$ is $2 X 2$ null matrix) \\
$\Longrightarrow$ $D$ $mod$ $p$ == $0$ \\\\
Hence, every entry of $D$ is muliple of $p$, or conversely $p$ is factor of every entry of $D$. \\
Now, take any entry of $D$ \textit{(say x)}. Now, x is bounded by $m^{th}$ Fibonacci number which is bounded by $2^{m}$; \\
Now, any number $N$ has at most $logN$ factors hence x has atmost $m$ factors or atmost $n$ factors ( m $\leq$ n). \\ 
Let say we choose $p$ uniformly randomly from range $[2,t]$ then probability that a factor of x was choosen was $\frac{n}{\pi(t)}$ \\
There are four elements of $D$ hence probability that p was a factor of all of the element $\leq$ 
$(\frac{n}{\pi(t)})^{4}$.\\ 
We want this probability to be less than $\frac{1}{n^{4}}$. \\
Hence, we want $\frac{n}{\pi(t)}$ to be less that $\frac{1}{n}$. \\ 
Putting $t$ = $n^{2}logn$ 
and $\pi(t)$ = $\frac{t}{logt}$ .\hfill{As done in class}   \\
We get this probability to be less $\frac{1}{n^{4}}$.\\\\
3. \\
\textbf{Notation} : \\\\
Here we assume that all operation are accompanied by a modulo operation by p which is chosen in the begining. \\ 
Let S me 2-D bit-matrix and $S(i,j)$ denotes the element of $i_{th}$ row and $j_{th}$ column. \\
$Prefixproduct(i,j)$ = Product of all the $M(S(k,j))$ s.t. 1 $\leq$ k $\leq$ i. \\
i.e. $Prefixproduct(i,j)$ = $M(S(1,j)) * M(S(2,j)) * ... * M(S(i,j))$\\
$Prefixproductinverse(i,j)$ = Matrix inverse of $Prefixproduct(i,j)$ \\
i.e. $Prefixproductinverse(i,j)$ = $M(S(i,j))^{-1} * M(S(i-1,j))^{-1} * ... * M(S(1,j))^{-1}$ \\
Now let us assume there is a matrix X of size $NXM$ then we have map this matrix to 2X2 matrix M(X) \\
We define to this map is follows: \\ 
If X is column matrix, cut this matrix into two submatrix by cutting at any row. Let U be upper column matrix and L be the lower one. \\
Then $M(X)$ = $M(U) * M(L)$ \\ 
If X is not a column matrix,cut this at any column, let L be the submatrix left of this line and R be the one right to it.
Then $M(X)$ = $M(L) * M(R)$ \\
Now we have defined the new mapping M, we can state out algorithm. \\
\textbf{Algorithm} : 
Let T be the text $nXn$ 2-D bit-matrix and P be the pattern $mXm$ 2-D bit-matrix.
First we show how to calculate $Prefixproduct(i,j)$ and $Prefixproductinverse$ for matrix T in $O(n^{2})$ time. \\
$Prefixproduct(1,j)$ and $Prefixproductinverse(1,j)$ will be simply the M of the corresponding element.
To calculate them for other rows just use the $Prefixproduct(i,j)$ and $Prefixproductinverse$ of just upper row. \\
Following pseudocode describes out algorithm : \\
for j = 1 to n  \\
\hspace*{1cm}	$Prefixproduct(1,j)$ = $M(T[1,j])$ \\ 
\hspace*{1cm}	$Prefixproductinverse(1,j)$ = $M(T[1,j])^{-1}$ \\
for i = 2 to n  \\ 
\hspace*{1cm} for j = 1 to n \\ 
\hspace*{2cm} $Prefixproduct(i,j)$ = $Prefixproduct(i-1,j)*M(T[i,j])$ \\
\hspace*{2cm} $Prefixproductinverse(i,j)$ = $M(T[i,j])^{-1}*Prefixproductinverse(i-1,j)$ \\ \\
Also we calculate $M(P)$ in $O(m^{2})$ time. \\ 
Initialize M(P) = $I_{2x2}$ \\
for i = 1 to m \\ 
\hspace*{1cm}for j = 1 to m \\  
\hspace*{2cm} $M(P)$ = $M(P) * M(T(j,i))$ \\\\
Now we have to calculate mapping of every submatrix of size m of matrix P. Let $M(P(i,j))$ is the mapping of submatrix of size m ending at (i,j) in P. $m$ $\leq$ $i,j$ $\leq$ $n$. We will calculate this mapping in $O(n^{2})$ time as follows : \\
First we calculate $M_c(i,j)$ which is product of all the $M(P[k,j])$
such that $i - m + 1$ $\leq$ $k$ $\leq$ $i$, $m$ $\leq$ $i$ $\leq$ $n$,$1$ $\leq$ $j$ $\leq$ $n$ \\
Also we will calculate $M_c(i,j)^{-1}$ with it. 
We can calculate these both in $O(n^{2})$ as follows:\\
If $i$ = $m$ \\
\hspace*{1cm} $M_c(m,j)$ = $Prefixproduct(m,j)$ \\
\hspace*{1cm} $M_c(m,j)^{-1}$ = $Prefixproductinverse(m,j)$ \\
else \\
\hspace*{1cm} $M_c(i,j)$ = $Prefixproductinverse(i-m,j)*Prefixproduct(i,j)$ \\
\hspace*{1cm} $M_c(m,j)^{-1}$ = $Prefixproductinverse(i,j)*Prefixproduct(i-m,j)$ \\ 
Now with help of this we can calculate $M(P(i,j))$ in $O(n^{2})$ as follows : \\
if $j$ = $m$ \\
\hspace*{1cm}	Initialize $M(P(i,j))$ = $I_{2x2}$ \\ 
\hspace*{1cm}	for k = 1 to m  \\ 
\hspace*{2cm}		$M(P(i,j))$ = $M(P(i,j)) * M_c(i,k)$ \\
else \\
\hspace*{1cm} $M(P(i,j))$ = $M_c(i,j-m)^{-1}*M(P(i,j-1))*M_c(i,j)$\\\\
For $n-m+1$ elements we take $O(m)$ time and for rest of the elements we take $O(1)$ time. Hence total time = $O(m*(n-m+1))$ + $O((n-m+1)*(n-m+1))$ = $O(n^{2})$\\ 
Hence we just go at each i,j s.t m $\leq$ i,j $\leq$ n,compare M(T) with $M(P(i,j))$ in $O(1)$ time which take overall $O(n^2)$




\section*{2. How well did you internalize the proof of Chernoff bound ?} 
1. Probability that number of coin tosses required to get n heads is greater than $2n(1 + \delta)$ is same as Probability of getting less than or equal $n-1$ heads out $2n(1 + \delta) - 1$ coin tosses. \\ 
For given geometrically distributed random variable, we know that expected value is $\frac{1}{p}$ which is given $2$ , hence $p$ = $\frac{1}{2}$. \\
\[
	X_i^{`} =\begin{cases}
		1  \hspace{1cm}   $if head is obtained at $i_{th}$ iteration$ \\
		0  \hspace{1cm}   $otherwise$ 
		\end{cases}
\] %
$E[X_i^{`}]$ = $\frac{1}{2}$ * 1 + $\frac{1}{2}$ * 0 = $\frac{1}{2}$. \\\\
$X^{`}$ = $\sum_{i=1}^{2n(1 + \delta) - 1}X_i^{`}$ \\\\
$E[X^{`}]$ = $n(1+\delta) - \frac{1}{2}$ \\\\
Assuming $n \gg \frac{1}{2}$, we get $E[X^{`}]$ =  $n(1+\delta) $ \\\\
P($X^{`} \leq (n-1)$) \\\\
= P($X^{`} \leq (n-1)\frac{E[X^{`}]}{(1 + \delta)n}$)\\\\
= P($X^{`} \leq (1 - \frac{n\delta+1}{n(1+\delta)})*E[X^{`}])$ \\\\
$\leq$ P($X^{`} \leq (1 - \frac{\delta}{\delta + 1})*E[X^{`}])$ (As $\frac{n\delta+1}{n(1+\delta)}$ $\textgreater$ $\frac{\delta}{\delta + 1}$ )\\\\
= P($X^{`} \leq (1 - \delta^{`})*E[X^{`}] $)
By letting $\delta^{`}$ = $\frac{\delta}{\delta + 1}$ \\\\ 
$\leq$ $(\frac{e^{-\delta^{`}}}{(1 - \delta^{`})^{1 - \delta^{`}}})^{E[X^{`}]}$\\\\
Putting value of $\delta^{`}$, we get \\\\ 
$\leq$ $((1 + \delta)e^{-\delta})^{n}$ \\\\
2.
According to Markov's Inequality \\\\
P(X $\geq$ a) $\leq$ $\frac{E[X]}{a}$ if X($\omega$) $\geq$ 0 $\forall$ $\omega$ $\in$ $\Omega$ \\\\
$\because$ X($\omega$) $\in$ $\mathbb{N}$ \\\\ 
P(X $\geq$ $(1 + \delta)\mu$) \\\\
= P($e^{tx}$ $\geq$ $e^{(1 + \delta)t\mu}$) $\leq$ $\frac{E[e^{tx}]}{e^{(1 + \delta)t\mu}}$\\\\
E[$e^{tX}$] \\\\
= E[$e^{tX_1 + tX_2 + tX_3 ... + tX_n}$] \\\\
=  $\prod_{i=1}^{n} E[e^{tX_i}]$ \\\\
E[$e^{tX_i}$] = $\frac{e^{t}}{2}$ + $\frac{e^{2t}}{4}$ + $\frac{e^{3t}}{8}$ .... 
= $\frac{e^{t}}{2(1-\frac{e^{t}}{2})}$
= $\frac{e^{t}}{2 - e^{t}}$ \\
E[$e^{tX}$] = $\prod_{i=1}^{n} E[e^{tX_i}]$ = $(\frac{e^{t}}{2 - e^{t}})^{n}$ \\\\
Now $\mu$ = $2n$ \\\\
P(X $\geq$ $2n(1 + \delta)$) $\leq$ $(\frac{e^{t}}{(2 - e^{t})e^{2t(1+\delta)}})^{n}$\\\\
Now differentiating w.r.t $t$ to get the $minima$ we get \\\\
P(X $\geq$ $2n(1+\delta)$) $\leq$ $(\frac{(1+\delta)^{2(1+\delta)}}{(1+2\delta)^{1+2\delta}})^{n}$ \\\\
3. \\
\textbf{Claim} : Bound obtained from 2nd part is better \\\\ 
\textbf{Proof} : Bound obtained in 1st part \textit{(say b1)} = $(\frac{1+\delta}{e^{\delta}})^{n}$ \\\\ 
Bound obtained in 2nd part \textit{(say b2)} = $(\frac{(1+\delta)^{2(1+\delta)}}{(1+2\delta)^{1+2\delta}})^{n}$ \\\\
To prove $\frac{b1}{b2}$ $\geq$ $1$
$\frac{b1}{b2}$ = $(\frac{(1+\delta)(1+2\delta)^{1+2\delta}}{((1+\delta)^{2(1+\delta)})e^{\delta}})^{n}$ \\\\
Taking $ln$ on both sides,we get \\\\
$n(ln(1+\delta) - \delta - 2(1+\delta)ln(1+\delta) + (2\delta + 1)*ln(1+2\delta))$ $\geq$ 0 \\\\
$(2\delta + 1)(ln(\frac{1 + 2\delta}{1 + \delta})) \textgreater \delta$ \\\\
$ln(\frac{1 + 2\delta}{1 + \delta})$ = $ln(1+\frac{\delta}{1+\delta})$ $\textgreater$ $\frac{\delta}{\delta + 1}$ - $\frac{\delta^{2}}{2(\delta + 1)^{2}}$ \\\\
LHS \textgreater $(2\delta + 1)(\frac{\delta}{\delta + 1} - \frac{\delta^{2}}{2(\delta + 1)^{2}})$
= $\frac{\delta(2\delta + 1)(\delta + 2)}{2(\delta + 1)^{2}}$
= $\frac{\delta(2\delta^{2} + 5\delta + 2)}{(2\delta^{2} + 4\delta + 2)}$
$\textgreater \delta $ \\
Hence proved. \\\\
\section*{3. Estimating biasness of a coin }
Strategy:
We toss the coin N times, Let $H$ be the random variable denoting the number of heads in N coin tosses. We report $\widetilde{p}$ to be $\frac{H}{N}$. \\
Now, given $\epsilon$, $\delta$ and $a$, we will estimate the value of N so that the following probability hold. \\
Pr[$|p - \widetilde{p}|$ $\textgreater$ $p\epsilon$] $\textless$ $\delta$ \\\\
$\Longrightarrow$ Pr[$\widetilde{p}$ $\textless$ $p(1 - \epsilon)$]
+ Pr[$\widetilde{p}$ $\textgreater$ $p(1 + \epsilon)$] $\textless$ $\delta$ \\\\
$\Longrightarrow$ Pr[$N\widetilde{p}$ $\textless$ $Np(1 - \epsilon)$]
+ Pr[$N\widetilde{p}$ $\textgreater$ $Np(1 + \epsilon)$] $\textless$ $\delta$ \\\\
$\Longrightarrow$ Pr[$H$ $\textless$ $E[H](1-\epsilon)$] + Pr[$H$ $\textgreater$ $E[H](1+\epsilon)$] $\textless$ $\delta$ \\\\
Now Pr[$H$ $\textless$ $E[H](1-\epsilon)$] + Pr[$H$ $\textgreater$ $E[H](1+\epsilon)$] \\\\ $\textless$ $e^{\frac{-N\epsilon^{2}p}{2}}$ + $e^{\frac{-N\epsilon^{2}p}{4}}$ \\\\ $\textless$ $2e^{\frac{-N\epsilon^{2}p}{4}}$ \\\\ $\textless$ $2e^{\frac{-N\epsilon^{2}a}{4}}$ \\\\
If we have this term to be less than $\delta$ then our Probability condition will hold.\\\\ 
$2e^{\frac{-N\epsilon^{2}a}{4}}$ $\textless$ $\delta$ \\\\
$\Longrightarrow$
$\frac{-N\epsilon^{2}a}{4}$ $\textless$ $ln(\frac{\delta}{2})$
\\\\
$N$ $\textgreater$ $4\frac{ln(\frac{2}{\delta})}{a\epsilon^{2}}$

\end{document}